\documentclass{article}
\usepackage{CJKutf8}
\documentclass[12pt,a4paper]{article}
\usepackage[utf8]{inputenc}
\usepackage[vietnamese]{babel}
\usepackage{amsmath,amssymb}
\usepackage{enumitem}
\usepackage{multicol}
\usepackage{fancyhdr}
\usepackage{geometry}
\geometry{margin=2.5cm}

\pagestyle{fancy}
\fancyhf{}
\rhead{Đề thi trắc nghiệm HSK 1 (Bài 1–5)}
\lhead{Giáo viên: ChatGPT}
\cfoot{\thepage}

\title{\textbf{ĐỀ THI TRẮC NGHIỆM HSK 1 – BÀI 1–5}}
\author{}
\date{}

\begin{document}

\begin{CJK*}{UTF8}{gbsn}

\maketitle
\section*{Hướng dẫn làm bài}
\begin{itemize}
    \item Thời gian làm bài: 30 phút
    \item Chọn một đáp án đúng nhất cho mỗi câu.
\end{itemize}

\begin{multicols}{2}
\begin{enumerate}[label=\textbf{Câu \arabic*.}, leftmargin=*, itemsep=1.2em]

% --- TỪ VỰNG - DỄ (1-4) ---
\item Từ “你好” có nghĩa là gì?
\begin{enumerate}[label=A.]
\item Tạm biệt
\item Xin chào
\item Cảm ơn
\item Không có gì
\end{enumerate}

\item “妈妈” nghĩa là?
\begin{enumerate}[label=A.]
\item Bố
\item Con gái
\item Mẹ
\item Cô giáo
\end{enumerate}

\item “中国” là quốc gia nào?
\begin{enumerate}[label=A.]
\item Hàn Quốc
\item Trung Quốc
\item Nhật Bản
\item Việt Nam
\end{enumerate}

\item “老师” là từ chỉ ai?
\begin{enumerate}[label=A.]
\item Bác sĩ
\item Học sinh
\item Cảnh sát
\item Giáo viên
\end{enumerate}

% --- NGỮ PHÁP - DỄ (5-8) ---
\item Câu nào sau đây đúng ngữ pháp?
\begin{enumerate}[label=A.]
\item 我是学生。
\item 是我学生。
\item 学生我是。
\item 学生是我。
\end{enumerate}

\item Trong câu “你好吗?”, từ “吗” dùng để làm gì?
\begin{enumerate}[label=A.]
\item Dùng để hỏi
\item Dùng để phủ định
\item Dùng để chào hỏi
\item Dùng để khen ngợi
\end{enumerate}

\item Câu trả lời phù hợp cho “你叫什么名字?” là:
\begin{enumerate}[label=A.]
\item 我不叫
\item 我是学生
\item 我叫李华
\item 你叫什么
\end{enumerate}

\item Chọn trật tự từ đúng cho câu “Tôi là người Trung Quốc”:
\begin{enumerate}[label=A.]
\item 是中国人我。
\item 我中国人是。
\item 我是中国人。
\item 中国人我是。
\end{enumerate}

% --- ĐỌC HIỂU - DỄ (9-12) ---
\item Đọc đoạn sau: \\
“我叫王明。我是中国人。” \\
Câu nào đúng?
\begin{enumerate}[label=A.]
\item 他是学生
\item 他不是中国人
\item 他姓王
\item 他叫王明
\end{enumerate}

\item Đọc đoạn sau: \\
“她是老师。” \\
Câu “她” chỉ ai?
\begin{enumerate}[label=A.]
\item Anh ấy
\item Tôi
\item Cô ấy
\item Họ
\end{enumerate}

\item Câu nào diễn đạt đúng: “Tôi không phải là giáo viên”?
\begin{enumerate}[label=A.]
\item 我是老师。
\item 我不是老师。
\item 老师不是我。
\item 是我老师。
\end{enumerate}

\item Đọc đoạn sau: “他是学生,他不老师。” Câu này sai ở đâu?
\begin{enumerate}[label=A.]
\item “他” sai
\item “学生” sai
\item “不” sai vị trí
\item Không sai
\end{enumerate}

% --- TỪ VỰNG - TRUNG BÌNH (13-16) ---
\item “名字” có nghĩa là:
\begin{enumerate}[label=A.]
\item Quốc tịch
\item Tên
\item Nghề nghiệp
\item Gia đình
\end{enumerate}

\item “谁” dùng để hỏi về:
\begin{enumerate}[label=A.]
\item Khi nào
\item Ở đâu
\item Làm gì
\item Ai
\end{enumerate}

\item “他是我朋友。” nghĩa là gì?
\begin{enumerate}[label=A.]
\item Anh ấy là học sinh của tôi
\item Anh ấy là bạn tôi
\item Tôi là bạn anh ấy
\item Tôi không quen anh ấy
\end{enumerate}

\item Chọn từ điền vào chỗ trống: “我____中国人。”
\begin{enumerate}[label=A.]
\item 吗
\item 是
\item 不
\item 名字
\end{enumerate}

% --- NGỮ PHÁP - TRUNG BÌNH (17-20) ---
\item Câu nào là câu hỏi?
\begin{enumerate}[label=A.]
\item 你是学生。
\item 他不是老师。
\item 你好吗?
\item 我叫王芳。
\end{enumerate}

\item “你是哪国人?” nghĩa là:
\begin{enumerate}[label=A.]
\item Bạn là ai?
\item Bạn tên gì?
\item Bạn là người nước nào?
\item Bạn có khỏe không?
\end{enumerate}

\item “我姓张” nghĩa là:
\begin{enumerate}[label=A.]
\item Tôi là người họ Zhang
\item Tôi tên là Zhang
\item Tôi không biết họ
\item Tôi không phải họ Zhang
\end{enumerate}

\item Câu phủ định đúng là:
\begin{enumerate}[label=A.]
\item 他不是学生。
\item 是他不是学生。
\item 他学生不是。
\item 不学生他。
\end{enumerate}

% --- ĐỌC HIỂU - TRUNG BÌNH (21-24) ---
\item Đọc đoạn sau: \\
“我叫李雷。我妈妈是老师。” \\
Ai là giáo viên?
\begin{enumerate}[label=A.]
\item Tôi
\item Bố tôi
\item Mẹ tôi
\item Bạn tôi
\end{enumerate}

\item Đọc đoạn: \\
“他是学生,他叫小明。” \\
Từ “他” chỉ ai?
\begin{enumerate}[label=A.]
\item Cô ấy
\item Tôi
\item Anh ấy
\item Em bé
\end{enumerate}

\item Đọc đoạn: \\
“我是中国人,她也是中国人。” \\
Từ “也” có nghĩa là gì?
\begin{enumerate}[label=A.]
\item Nhưng
\item Cũng
\item Là
\item Vì
\end{enumerate}

\item Đoạn: \\
“你叫什么名字?” \\
Chọn câu trả lời phù hợp:
\begin{enumerate}[label=A.]
\item 我不去
\item 我不是老师
\item 我叫林娜
\item 是学生
\end{enumerate}

% --- TỪ VỰNG - KHÓ (25-26) ---
\item Từ nào KHÔNG xuất hiện trong 5 bài đầu?
\begin{enumerate}[label=A.]
\item 学生
\item 医生
\item 老师
\item 朋友
\end{enumerate}

\item Từ nào là đại từ nhân xưng số nhiều?
\begin{enumerate}[label=A.]
\item 他
\item 我
\item 我们
\item 她
\end{enumerate}% --- NGỮ PHÁP - KHÓ (27-28) ---
\item Câu nào dùng sai từ “吗”?
\begin{enumerate}[label=A.]
\item 你好吗?
\item 你是老师吗?
\item 他是中国人吗?
\item 我吗是学生。
\end{enumerate}

\item Chọn câu có từ “也” dùng đúng:
\begin{enumerate}[label=A.]
\item 他也学生。
\item 他是学生也。
\item 他也是学生。
\item 学生他也。
\end{enumerate}

% --- ĐỌC HIỂU - KHÓ (29-30) ---
\item Đọc: \\
“我叫小红,我不是学生,我是老师。” \\
Câu nào đúng?
\begin{enumerate}[label=A.]
\item 小红是学生
\item 小红不是老师
\item 小红是老师
\item 小红是朋友
\end{enumerate}

\item Đọc: \\
“他叫王明,是中国人。” \\
Từ “是” thể hiện:
\begin{enumerate}[label=A.]
\item Phủ định
\item Khẳng định
\item Nghi vấn
\item Kết thúc
\end{enumerate}

\end{enumerate}
\end{multicols}
\end{CJK*}
\end{document}